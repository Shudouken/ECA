\documentclass[12pt]{scrartcl}
\usepackage[utf8]{inputenc}
\usepackage[english,ngerman]{babel}
\usepackage[pdfusetitle]{hyperref}
\usepackage{graphicx}
\usepackage{calc}
\usepackage{caption}
\usepackage{fancyhdr,multicol,multirow,paralist,graphicx,array,enumitem}
\usepackage{amsmath}
\usepackage{float}
\usepackage{color}
\usepackage[dvipsnames]{xcolor}
\addto\captionsngerman{\renewcommand{\figurename}{Figure}}
\KOMAoptions{parskip=half+,paper=a4,twocolumn=false,DIV=12}
\fancypagestyle{eca-style}{
\fancyhf{}
\fancyhead[L]{ECA Summary}
\fancyhead[R]{\thesection}
\fancyfoot[C]{\thepage}
\renewcommand{\headrulewidth}{0.4pt}
\renewcommand{\footrulewidth}{0.4pt}}
\fancypagestyle{index-style}{
\fancyhf{}
\fancyhead[L]{ECA Summary}
\fancyhead[R]{Index}
\fancyfoot[C]{\thepage}
\captionsetup{width=0.5\textwidth}
\renewcommand{\headrulewidth}{0.4pt}
\renewcommand{\footrulewidth}{0.4pt}}
\setdescription{itemsep=0pt}
\makeatletter
\newenvironment{minipagespace}{
\begin{minipage}
}{
\end{minipage}
\vspace{10pt}
}
\makeatother
%\setenumerate{nosep}
\setcounter{tocdepth}{3}
\title{ECA Summary}
\date{2016}

\newcommand{\img}[3]{
\begin{figure}[H]
	\centering
	\includegraphics[#2]{img/#1}
	\captionsetup{width=0.8\textwidth, justification=centering}
	\caption{#3}
\end{figure}
}

\newcommand{\eq}[1]{
\begin{equation*}
#1
\end{equation*}
}

\begin{document}

%\pagestyle{index-style}
%\tableofcontents{}
%\newpage

%%%%%%%%%%%%%%%%%%%%%%%%%%%%%%%%%%%%%%%%%%%%%%%%%%%%%%%%%%%%%%%%%%%%%%%%%%%%%%%%%%%%%%%%%%%%

\pagestyle{eca-style}

\section{Enumerating vs. Counting}

\subsection{Permutation of Letters}

{\bf Example:} How many words can we make out of the letters A B C using each letter once?

\begin{center}
\begin{inparaitem}
\item ABC \qquad
\item BAC \qquad
\item CAB
\end{inparaitem}

\begin{inparaitem}
\item ACB \qquad
\item BCA \qquad
\item CBA
\end{inparaitem}
\end{center}

When we list all objects as above we call it {\bf enumeration}, whereas {\bf counting} is only concerned with the total number of objects. If we consider the example above, how many words would be possible for A B C D?

It's best to find a formula, as using it is a very efficient way to count Objects. For $n = 4$ letters we end up with $24$ permutations.

The formula for the amount of different words with $n$ letters is $n!$

\subsection{Points in Convex Position}

How many crossing-free spanning paths exist for $n$ points on convex position?

\img{crossing_free_spanning_path}
    {scale=1.5, trim=220 550 430 125}
    {An example illustrating one possibility of a spanning path for $n=9$ points}

For $n=1$ points the definition of the spanning path is unclear, in some cases it is considered as path with the size $1$ and in others with size $0$.

Let's look at some examples for $n > 1$ and try to determine a suitable formula. 

\img{crossing_free_spanning_path_enumeration}
    {scale=1, trim=270 340 200 125}
    {Enumeration of crossing free spanning paths up to $n=4$}

As we can see from this example, enumeration can become a tedious and error prone task very fast. Can you list all paths for $n=5$?

It is better to abstract the problem and find an inductive solution. When constructing the path we start with a point, and from it we only see two immediate choices. After one of those points is added, we have two choices again. This goes on for a while until $n-2$.

\eq{\underbrace{2 \cdot 2 \cdot 2 \cdots 2 \cdot 2}_{n-2 \text{ times}} = 2^{n-2}}

Now in order to construct all paths we need to start at all possible points, when we do that however a double count occurs.

\eq{n \cdot 2^{n-2} \Rightarrow \frac{n \cdot 2^{n-2}}{2} \Rightarrow n \cdot 2^{n-3} \text{ for } n \ge 2 }

We can use this formula to find the number of crossing-free spanning paths for $n=5$, which gives us $5 \cdot 2^2 = 20$.

\img{crossing_free_spanning_path_n_5}
    {scale=1, trim=200 425 200 125}
    {Another method of enumeration, do not list similar objects}

\end{document}
